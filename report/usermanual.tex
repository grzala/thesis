\chapter{User Manual \label{chap:usermanual}}
The system is designed to be cross platform, however it is advised to use Windows as it is the only system on which the project was thoroughly tested.

\section{Requirements and installation \label{sec:requirements}}
Before using the software, several requirements must be fulfilled.

\subsection{Software and Libraries}
\subsubsection{Required software}
\begin{itemize}
	\item Python 2.7
	\item Pip
	\item Blender version 2.79 or newer
\end{itemize}

\subsubsection{Libraries}
Only for Linux systems: Download and install library python-dev (or python-devel)

Following python libraries must be installed:
\begin{itemize}
	\item setuptools
	\item watson-developer-cloud
	\item wget
\end{itemize}
\noindent The packages can be installed in bulk by running:

\indent `path\_to\_python/Scripts/pip install -r requirements`

\noindent in the main directory. As Blender usually comes with its own Python environment, you will need to run the command:

\indent `path\_to\_blender/version/python/Scripts/pip install -r requirements`

\noindent To install the libraries for Blender.

\subsection{Install Generator as Blender Addon \label{sec:installaddon}}
The following steps are optional. Installing the module as addon will enable the module to be permanently incorporated into Blender.

\noindent To install the addon:
\begin{itemize}
	\item Open Blender
	\item Open user settings (File > User preferences)
	\item Open Add-ons tab
	\item On the bottom panel press `install add-on from file`
	\item Navigate to `project folder > generator` and double click on `addon.py`
	\item In the search box on the top right corner type `NLPanim` - a greyed-out item called `Object: NLPAnim` should appear
	\item Check the box next to the item
	\item Press `Save user settings` in the bottom left corner
	\item Exit user preferences
\end{itemize}

\noindent The module is now installed as an add-on. In the 3D-view (default view) on the bottom of the right-hand side menu there should appear a tab called `Generate NLP anim` (figure ~\ref{fig:installedaddon}).
	
\begin{figure}[!ht]
	\centerline{\fbox{\includegraphics[width = 30em]{img/appendix/installedaddon.png}}}
	\caption{The generator addon menu}\label{fig:installedaddon}
\end{figure}

\section{Generating animations}
\noindent Now that all the requirements are met, the animation can be generated. There are two main steps to generating. Firstly, use the analyzer module to create a JSON file instructions for the generator. Secondly, use the generator to assemble the animation.

\subsubsection{Analyze script \label{sec:analyze-script}}
\noindent The input script must resemble the script in figure ~\ref{inputscript2}

\begin{figure}[!ht]
	\centerline{\fbox{\includegraphics[width = 30em]{img/script.png}}}
	\caption{An example of system's input}\label{fig:inputscript2}
\end{figure}

\noindent The characters must be specified after 5 tabs. The dialogue text is specified after 3 tabs. The file must end with an `ENDSCRIPT` with no indentation. For reference please refer to either:
\begin{itemize}
	\item Example script files in folder `movies`
	\item Movie scripts hosted on \url{www.imsdb.com}
\end{itemize}

\noindent The script now can be process using the following command:

\indent `python analyzer/script\_analyze.py path\_to\_dialogue\_script db/animation.db`

\noindent The program will output a file called `scene.json`. This file is used to generate the animation.

\subsubsection{Generate Animation}
\noindent If you did not follow section ~\ref{sec:installaddon}, please follow these steps:
\begin{itemize}
	\item Open the file `rendered.blend` with Blender
	\item Click `Run Script` on the bottom of the scripting view (figure ~\ref{fig:withoutaddon})
	\item The tab called `Generate NLP anim` should appear on the right hand side menu of the 3D view
	\item Click on the `Generate NLP anim` tab
\end{itemize}

\begin{figure}[!ht]
	\centerline{\fbox{\includegraphics[width = 30em]{img/appendix/withoutaddon.png}}}
	\caption{Running the generator without installing it as an add-on}\label{fig:withoutaddon}
\end{figure}

Please follow these steps only if you installed the generator as an add-on:
\begin{itemize}
	\item Open Blender
	\item Press `File > Save` and save the file in a preferred location
	\item \textbf{Important:} close Blender and reopen the saved file
	\item Open `Generate NLP anim` tab
\end{itemize}

\noindent After finishing the above instructions, do the following to finalize the animation:
\begin{itemize}
	\item Press `Clean` at the top of the menu
	\item Choose the directory `mocap` for `Animations Directory`
	\item Choose the directory `models` for `Models Directory`
	\item Choose the file `db/animation.db` for `Database`
	\item For `Dialogue` choose the `scene.json` file you created when following the section ~\ref{sec:analyze-script}
	\item press `Prepare` 