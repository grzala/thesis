\chapter{Requirements Specification\label{chap:requirements}}

\noindent This section describes functional and non-functional requirements of the proposed software.


\section{Functional Requirements}

\noindent {\bf FR1 Analysis of a semi structure script}

\noindent The software takes a semi structured script as input. The structure of a script must resemble a structure of a movie script. The script provides 3 kinds of information - characters involved in a scene, dialogue lines and actions performed by characters during the scenes. The software must be able to extract actions and emotions from the script.

\bigskip

\noindent {\bf FR2 Find relevant animation clip}

\noindent The software must take information about character's emotions and actions and choose animation clip that best represent's characters behaviour.

\bigskip

\noindent {\bf FR3 Assemble final scene}

\noindent The software takes information generated by other modules and outputs an animated scene. The outputted scene must be fully editable, enabling various adjustments before rendering.


\section{Non-Functional Requirements}

\begin{itemize}
\item The user should be able to use the software with a custom motion-capture database (Usability).
\item The software is designed to automatically create big amounts of animated scenes. The time of generating a scene is not a high priority, but it must be reasonable (peformance).
\item The software must support common animation file formats (fbx, bvh) (portability).
\end{itemize}
