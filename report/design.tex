\chapter{Design And Architecture\label{chap:design}}

This chapter describes the design of the system. This includes both the underlying decisions of the system as well as the user interface design. The chapter also presents on a more detailed view of the system's architecture.

\section{System Design}

The system has to work in two major steps. The first step is to take semi-structured natural language script and analyze it. This means that the system must analyze each dialogue line and infer some emotional values from them. Using those emotional values and the length of the dialogue line, the system must find best matching animation clips from the motion capture database. The results of this step is a file that specifies which characters say which lines of dialogue and it also specifies which animated clips accompany those dialogue lines.

After this step is completed, the file must be intepreted into a final scene. The importer must be able to read the file, import required models and animations and generate the final scene. As every animation software and game engine is a little different, each of them would need a custom importer. Those would be very similar in principle, but differ slightly because of the implementation of given software. For my project I have created the importer for Blender. This is because Blender is open source and available to anyone, as well as I am the most familiar with this software.


\section{Emotions}

For my project I have decided to only use the following emotions:
\begin{itemize}
\item Joy
\item Anger
\item Fear
\item Sadness
\item Disgust
\item Surprise
\end{itemize}

These are six basic human emotions as specified by Paul Ekman and Wallace V. Friesen. For each animation and dialogue line the system assigns a value between 0 and 1 for each of those emotions in order to indicate whether a given emotion is present in an action/text and how intense that emotion is. Any combination of those emotions may be used to achieve more specific results.

This set of six basic emotions is generally agreed to be satisfiably accurate in most cases. A lot of existing software (such as IBM Watson Tone Analyzer) and lexicons use those exact emotions.


\section{architecture}

The architecture of the system is essentially a pipeline. The modules process resources and pass them onto the next module. They can be completely unaware of each other. This allows for a lot of flexibility and helps achieve some of the requirements. The emotion analysis API can be replaced with a different one, the user can use a custom motion capture database, and a different importer can be created if the user wishes to use software other than blender.


There are three main modules:
\begin{itemize}
\item Text Analysis
\item Animation Clip Matcher
\item Animation Generator
\end{itemize}

The Text Analysis module parses the text and analyses emotion using some API. It takes a script as an input and passess the parsed and analyzed text to the Animation Clip Matcher.

The Animation Clip Matcher uses the motion capture database to find the best animation clips to accompany the speech. The Animation Clip Matcher must take into account the emotion analysis of the text and find animations with similar emotional score. Another important constraint is the time of the animation. The animated clips have different lengths and a chosen clip must not be significantly longer than the spoken/read text. In case the animated clip is shorter, the Matcher must choose more than one matching clip. The Matcher outputs a JSON file which specifies all dialogue lines; it states which characters perform which emotional gesture actions and where is the animation file located.

The Animation Generator reads the JSON file and imports all needed animations. The user is now able to assign character model for each character and run the generator. The generator will assemble the animations in correct order, focus the camera on a currently speaking character and add subtitles. The output is an animated scene that can be either manually edited, exported to a file or rendered.

Python programming language is used to implement the modules. This is mainly due to the fact that Blender only supports extensions created using Python. For modules that do not rely on Blender, Python was chosen due to high development speed.


\section{Input}

The input of the system looks as follows:

The input is a file that represents a dialogue. Each character that participates in the scene must ble clearly stated. Each dialogue line must have a dialog line must have a character clearly associated with it. Character names are specified after five tabs. Dialogue Lines are specified after three tabs. Each file must end with "ENDSCRIPT" with no indentation.

I used this format as this format is often used to represent movie scripts. One can find hundreds of scripts saved in this format on The Internet Movie Script Database (\url{www.imsdb.com}).


\section{Database}

The motion capture database consists of two main parts: the files that contain the animated clips and a database that holds the metadata about the animations.

Most animation come from the Max Planck's Research Institute Emotional Body Motion Database. This database is a set of motion capture clips performed by actors. The clips are supposed to represent emotions in various settings. Most of these animations need minor changes in order to be compatible with my system (e.g. a lot of clips were recorded while the actor was sitting, so the leg keyframes must be removed to make the model stand). The animations are exported to FBX files (proprietary file format owned by Autodesk, widely supported by many different frameworks) one animated clip per file.

The database is implemented using SQLite. It is the perfect tool for this task as the database needs to be simple, relatively small and easily searchable. The animation metadata is held in a table that look like this FIGURE. The emotional values of each animated clip need to be manually adjusted. When all values are set to zero, it means that the clip carries no emotional impact (neutral).



\section{Matcher}

does some stuff. explain matching algorithm here.


\section{Animation Generator (Importer)}


As a forementioned, in this project I have used Blender for creating the animation. The generator is a Blender addon script. The generator first reads the JSON file prepared by previous modules and loads required animation clips from files. Each action is assigned to a corresponding character model and then pushed on a separate NLA strip FIGURE. Subtitles are added as Video Editor Sequence Strips spanning between frames that correspond to the animation FIGURE. The camera is animated to focus on a character that is currently talking.


\section{User Interface}

The Animation Generator module is the only module that features a Graphical UI. The module is embedded into Blender and uses extends Blender's interface. The UI is minimalistic and simple to use. It requires the user to specify an input file, as well as animation directory. Upon execution, the extension will pre-load the animations and determine characters involved in the scene. Now it is up to the user to specify which model is supposed to be used for each character. Upon pressing finalize, the animation will be created. From now on The user can continue to use Blender normally.

The JSON file must be created first outside of blender by running a script. This is because the modules need to be kept as separate as possible so that importers can be easily implemented for other platforms.

They key aspect of the interface is its simplicity as it allows users to generate relatively complicated animation sequences by essentially using two buttons, with no knowledge about animation and little Blender expertise required. The UI also provides flexibility as the user can decide upon which character models to use. The final UI is presented in FIGURE.





