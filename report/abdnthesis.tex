\documentclass[phd]{abdnthesis}

%% For citations, I would recommend natbib for its                          
%% flexibility, particularly when named citation styles are used, but                
%% it also has useful features for plain and those of that ilk.                      
%% The natbib package gives you the following definitons                             
%% that extend the simple \cite:                                                     
%   \citet{key} ==>>                Jones et al. (1990)                              
%   \citet*{key} ==>>               Jones, Baker, and Smith (1990)                   
%   \citep{key} ==>>                (Jones et al., 1990)                             
%   \citep*{key} ==>>               (Jones, Baker, and Smith, 1990)                  
%   \citep[chap. 2]{key} ==>>       (Jones et al., 1990, chap. 2)                    
%   \citep[e.g.][]{key} ==>>        (e.g. Jones et al., 1990)                        
%   \citep[e.g.][p. 32]{key} ==>>   (e.g. Jones et al., p. 32)                       
%   \citeauthor{key} ==>>           Jones et al.                                     
%   \citeauthor*{key} ==>>          Jones, Baker, and Smith                          
%   \citeyear{key} ==>>             1990                                             

\usepackage[round,colon,authoryear]{natbib}
\setlength{\bibsep}{0pt}
\bibliographystyle{apalike}

\usepackage[T1]{fontenc}

\title{How to use abdnthesis.cls}
\author{Timothy J.\ Norman}
% IMO this is a bit silly, but some like to include these. To remove,
% delete this declaration and remove the option from the
% \documentclass definition above.
%\qualifications{PhD, Computer Science, University College London, 1997\\%            
%BEng (Hons.) Electrical and Electronic Engineering, The University of Wales, Swansea, 1992}
\school{Department of Computing Science}

%%%% In the final submission of a thesis, this should only be the year
%%%% of submission.  However, it is useful to use \date{\today} for drafts so that
%%%% they don't get mixed up.
    
\date{2010}

%% It is useful to split the document up as chapters and include
%% them.  LaTeX will sort out all the numbering and cross-referencing
%% for you --- if you run it enough times!

%% If you want to include only a couple of chapters then use the
%% \includeonly{} command with a list of the file/chapter names that
%% you wish to include.  NB, this must be in the preamble.

\includeonly{introduction,faq}

\def\sfthing#1#2{\def#1{\mbox{{\small\normalfont\sffamily #2}}}}

\sfthing{\PP}{P}
\sfthing{\FF}{F}

%% This will make sure that all cross-references are correct (including
%% references to those file not included) but will produce a dvi
%% file with only those files/chapters you specify included.

\begin{document}

%%%% Create the title page and standard declaration.

\maketitle
\makedeclaration

%%%% Then the abstract and acknowledgements

\begin{abstract}
  An expansion of the title and contraction of the thesis.
\end{abstract}

\begin{acknowledgements}
  Much stuff borrowed from elsewhere
\end{acknowledgements}

%%%% It should have a table of contents, but delete the other two as
%%%% necessary.

\tableofcontents
%\listoftables
%\listoffigures

\chapter{Introduction\label{chap:introduction}}

Each chapter heading is typeset in this way --- this in an integral
part of the style, so if you don't like it abdnthesis.cls may not be
for you. However, do feel free to modify the .cls file to your needs.

\section{Defaults}

\begin{itemize}
\item oneside --- assuming single sided printing.

\item onecolumn --- \LaTeX\ will give you an error if you try to use
  the twocolumn option, not that anyone would contemplate this for a
  thesis.

\item 11pt --- this works best with the text height and width on A4
  paper.

\item 1.5 line spacing --- looks much better than double spacing.

\item Times Roman font --- for both text and maths with the exception
  of mathcal, but see below for options.
\end{itemize}

\section{Options}

\begin{itemize}

\item These are mutually exclusive options that are used to specify
 the type of degree that the thesis is to be submitted in partial
 fulfilment of the requirements of:

\begin{itemize}
\item phd or PhD -- the default.
\item mphil or MPhil -- for Master of Philosophy Theses.
\item msc or MSc -- for MSc project reports.
\item bsc or BSc -- for BSc project reports.
\end{itemize}

\item Two self-explanatory options for changing the line spacing from
 the default 1.5.

\begin{itemize}
\item singlespace
\item doublespace
\end{itemize}

\item titlebox --- this option ensures that the title of the document
 fits within the window on the standard departmental BSc or MSc
 project report front cover.
 
\item twoside --- this option is if you wish to print your thesis
  using a double sided printer.  Note that the University regulations 
  do not permit the submission of theses printed double sided.

\item cmmath --- this action changes the font used in math mode to be
  Computer Modern. The default is for all math mode fonts to be times
  with the exception of the mathcal font. If you want to also use
  times for mathcal, you need to comment out these two lines in the
  abdnthesis.cls file:
\begin{verbatim}
        \SetMathAlphabet{\mathcal}{normal}{OMS}{cmsy}{m}{n}
        \SetMathAlphabet{\mathcal}{bold}{OMS}{cmsy}{b}{n}
\end{verbatim}

\item cmall --- if you want to go with Computer Modern for both
  text and maths, use this option.
\end{itemize}

There is also an \emph{optional} command for including prior
qualifications within the title page; some like to do this. Prior to
version 2.3 (2013/08/11) \LaTeX\ would give an error if this was not
declared, even if you didn't want this information on the title
page. This meant that you would have to have made the declaration
\verb+\qualifications{}+, which was an ugly solution. Now, ``empty''
declaration can simply be omitted.




\chapter{Frequently asked questions\label{chap:faq}}

In addition to the information provided in
chapter~\ref{chap:introduction}, here are some brief notes on
references (see section~\ref{sec:references}) and figures (see
section~\ref{sec:figures}).

\section{References\label{sec:references}}

You can, of course, use any referencing style you like such as
\verb+plain+.  The \verb+natbib+ package, however, allows you to do
this with named style citations:

\begin{tabbing}
\hspace{3in} \= \kill
\verb-\citet{key}- \> Jones et al. (1990) \\
\verb-\citet*{key}- \> Jones, Baker, and Smith (1990) \\
\verb-\citep{key}- \> (Jones et al., 1990) \\
\verb-\citep*{key}- \> (Jones, Baker, and Smith, 1990) \\
\verb-\citep[chap. 2]{key}- \> (Jones et al., 1990, chap. 2) \\
\verb-\citep[e.g.][]{key}- \> (e.g. Jones et al., 1990) \\
\verb-\citep[e.g.][p. 32]{key}- \> (e.g. Jones et al., p. 32) \\
\verb-\citeauthor{key}- \> Jones et al. \\
\verb-\citeauthor*{key}- \> Jones, Baker, and Smith \\
\verb-\citeyear{key}- \> 1990 \\
\end{tabbing}

\section{Figures\label{sec:figures}}

To include an encapsulated postscript or PDF file (depending on whether you're using \LaTeX or PDF\LaTeX) as a figure, do
something like the following.  Note, to ensure correct
cross-referencing, it is best to include the figure label within
the caption definition.  \emph{Note that the graphicx package 
is already loaded and used to include the
University crest on the title page.}

\begin{verbatim}
\begin{figure}
 \begin{center}
   \includegraphics{myfigure.pdf}    
   \caption{This is my figure.\label{fig:mylabel}}
 \end{center}
\end{figure}
\end{verbatim}

\section{Frequently used symbols\label{sec:fus}}

In \LaTeX\ documents where you want to use a modality or some text consistently in normal text and in equation environments it is often difficult to remember to typeset the text consistently or time-consuming to keep typing in the environment. It may be a good idea to define something like the following in the preamble (i.e.\ before \verb+\begin{document}+):

\begin{verbatim}
\def\sfthing#1#2{\def#1{\mbox{{\small\normalfont\sffamily #2}}}}

\sfthing{\PP}{P}
\sfthing{\FF}{F}
\end{verbatim}

Then use it in text or math mode. In all cases it looks the same; e.g.\\
\verb+\PP\ refers to something, and other things are \FF; $\Phi = \PP\cup\FF$+\\
is typeset as:

\PP\ refers to something, and other things are \FF; i.e.\ $\Phi = \PP\cup\FF$

Note that you need to put ``\textbackslash'' after the command if you want a normal space after it.

\include{literature-survey}
\chapter{Design\label{chap:design}}


\include{implementation}
\chapter{Evaluation \label{chap:eval}}
This chapter describes how the system was evaluated and explains why such a method of evaluation was chosen. The evaluation aims to determine whether the created system was successful or not and due to what reasons. 

\section{Problems with evaluation}
Evaluation of animation is quite problematic in nature. The main issue is, that how an animation is perceived is very subjective - it is near impossible to computationally determine whether an animation looks natural, whether an animation goes in pair with the speech and if the whole `feel` of the dialogue scene is good. It is then important for the animation to be judged by people. I myself however am not fit to do that - I have spent far too much time creating the animated scenes and watching both failed and successful results. Because of that I am obviously seriously biased and cannot judge the animations from an objective standpoint.


\section{Survey}
In order to account for my bias, I have decided to use interview people about the generated animations. People who have not had a part in the development of the software can provide valuable opinions.


\subsection{Methodology}
People may find it hard to provide an opinion about the quality of an animation if they have nothing to base their answer upon (especially if they are not animators or developers). There needs to be some sort of a benchmark, or a control sample to enable people to judge the animations more efficiently.

To account for that I have chosen the following approach: I would first find a dialogue scene in a relatively popular, successful game and recreate it using the developed software. The participants would be shown both the original and recreated scene and describe how they compare - they would be asked to decide whether using the software has improved or decreased the quality of the scene. 

\medskip
\subsubsection{Preparation}
I have chosen three games from which I extracted the dialogue scenes. I did not want to extract a few dialogue scenes from one game in order to provide variety, but I also did not want to use more than three games as this would make the questionnaire too long and make the participants loose interest or become confused.

The three games are all of the RPG genre and differ by their year of release - I have used a relatively old game, relatively new game and a state-of-the-art game - for if the software is deemed to produce animation of lower quality than the state-of-the-art game, comparing it to an old game may help decide whether the software simply needs improvements, or if the approach itself is wrong.

Not simply any game could have been chosen. In my project I have to rely on EBMD to generate the animations - and the motion capture data provided by the database is not suitable for any game. For example, if a game features a lot of soldier to soldier military interactions, it is expected that the characters will behave in a more serious, less expressive manner (I evaluate on that in section ~\ref{sec:otherfindings}). The EBMD does not supply animations that would be useful such a situation and using a wrong game would cause unnecessary confusion of the participants. The chosen games must feature relatively relaxed, natural conversations. This certainly tightens the pool of games and scenes usable for the evaluation, but I believe that the chosen games are a rather representative sample of the RPG genre.

The chosen games are:
\begin{itemize}
	\item Fallout 3 (released in 2008)
	\item Fallout 4 (released in 2015)
	\item Horizon: Zero Dawn (released in 2017)
\end{itemize}

The dialogues from the games were chosen randomly, however there were some constraints that limited the choice:
\begin{itemize}
	\item The scene cannot feature more than two characters (The software was designed with extendibility in mind, however at the moment it does not support more than two characters).
	\item The dialogue has to convey some emotions (dialogues which feature strictly neutral emotions are not truly representative of the software).
	\item The dialogue cannot be exposition dialogue (Why that is the case and what exactly is \textit{exposition dialogue} I explain in sections ~\ref{sec:evalotherfindings} and ~\ref{sec:otherfindings}).
\end{itemize}



Before conducting an interview it is important to know whether the interviewee has experience with games and animations and development. The answers of people knowledgeable in this area might differ from those provided by people new to this area and it might be helpful to be able to distinguish between them.

\medskip
\subsubsection{The questionnaire}
The questionnaire consists of four sections. The first section provides some details about what the software is trying to achieve and what is expected of the participant. This section also asks whether the participant considers themselves to be knowledgeable about games or animations.

The other sections all ask the same set of questions about different pairs of videos\footnote{The animations and videos can be accessed at \videoshost. A sample dialogue transcript with still images of the scene can be found in appendix section ~\ref{sec:samplescene}.}. At the start of each section, the participant will be shown two videos - an original and a recreated dialogue scenes. After watching both of them they will be asked a couple of questions that determines whether they find the videos realistic, whether the animations correspond to the conveyed gestures and whether they think that using the software improves the scenes. The participants are able to evaluate in more detail on the answers if they wish to do so. The questions that are designed to be answered on a linear scale that resembles the Likert scale\footnote{The Likert scale is a psychometric scale created by a psychologist Rensis Likert. It is commonly used in questionnaires in order to determine whether the participants agree or disagree with given statements.}.

The original dialogue scenes were modified in two ways. Firstly, apart from the in-game subtitles, another set of subtitles was added for accessibility purposes. The added subtitles were the same as the original subtitles, but bigger and easier to read. Secondly, the videos were muted. Audio is completely irrelevant to this project. I could not procure audio for the recreated scenes\footnote{Simply overlaying the audio from the original scene over the recreated scene is not a good solution either. The audio was not created with the recreated scene in mind and the recreated scene did not take audio into account. Because of that, the characters would perform body motion that puts stress on one part of a sentence while audio would put stress on a different part of a sentence. This makes the audio unusable with a scene as it would make the scene feel completely unrealistic.} therefore the recreated scenes are mute. People watching first a scene with audio and then a scene without audio would focus too much on that difference instead on the difference in animation. Therefore, it is best to remove any sound from all the videos.

The surveys were conducted as a one-on-one interview if possible. If the participants were unavailable or preferred not to do this personally, an online version of the questionnaire was provided. The participation in the questionnaire was fully voluntary. The participants were presented with a consent form before answering the questions and were provided with the details of the software and how the study is going to be conducted. No aspect of the software or the study was hidden from the participants and they were able to both request more details or quit any time they wished.

As mentioned before, the preferred approach was a supervised survey as it is important that the participants correctly understand what the project is about and what it is trying to achieve. The survey was targeted mostly at my colleagues (Bachelor computing science students), as certain level of computing science and application development expertise is desirable. That is because people who are unacquainted with prototyping and development might focus too much on how unpolished the animation is (since it is only a prototype and the animations are rather crude - there is no background in the scene, no audio and no texturing/graphics) than on the gestures and dialogue itself.

The full questionnaire can be found in the appendices in section ~\ref{sec:appquestionnaire}.

\medskip
\subsubsection{Objectives}
The survey has the following objectives:

\begin{addmargin}[2em]{2em}
	\noindent \textit{To test whether using the NLP approach for animation generation has created scenes that are more realistic and natural than the original scenes.}
	
	\medskip
	
	\noindent \textit{To test whether the animations generated using the NLP method are preferable to those created by traditional methods.}
	\medskip
	
	\noindent \textit{Determine how much manual adjustment is required before the animations generated by the software are satisfactory.}
\end{addmargin}



\section{Other findings \label{sec:evalotherfindings}}
Apart from the findings of the questionnaire, I also wish to describe my own realisations that I devised during the development of the software. I know that I cannot use them to decide the successfulness of the software because of my bias towards it, however, I think that there are a few aspects worth mentioning that will not be discovered by the questionnaire participants.

As I was testing the software many times and I tried to recreate many scenes from RPG games I have noticed that the software does not perform well if certain conditions are not satisfied. For example: the general `feel' of the game must match the gestures provided by the motion database, the software does not perform well with scenes that contain exposition or sarcasm. I will explain those issues in more detail in section ~\ref{sec:otherfindings}.





\chapter{Discussion \label{chap:discussion}}
This section discusses and evaluates the findings accumulated during the development and evaluation of the software.

\section{Survey results}

\subsection{Realism}
According to the results of the survey the realism of the generated animations is rather disappointing. The animations managed however to be relatively as realistic as an animation from a quite recent game, meaning they might be good enough to be used in a game (given how little it takes to generate them). The very important aspect is that the generated animation were not seen as realistic, but were seen as correct (matching speech with emotions and body language). It is interesting that people often preferred the recreated scenes even when they were less realistic than the originals. Interviewees dissatisfied with the recreated animations said that the movements of characters are `weird' and that the software is too sensitive, with characters overreacting to the actual situation. 

The survey successfully carried out its first objective, showing that the NLP approach did not produce animations more (or much less) realistic than the originals. There are a few explanations to that. The possible causes have been described in section ~\ref{sec:riskanal}.

The lack of realism was probably not caused by the emotion analysis. The interviewees rather agreed that the emotions in the speech match the body language. The results suggest that the problems lies within the movements itself. 

Both the motion capture data quality and the uncanny valley principle are possible causes of those results. It is impossible to say now which one is more important. A similar study which uses a different motion database could be helpful in determining that.

It is also possible that the results are skewed because of the crudeness of the entire scene. The models are rather bulky and poorly detailed, untextured, there is no background or audio. That is because the software focuses solely on the animation. However, the interviewees are not used to watching animation being developed and might instinctively see such a scene as less realistic and be unable to look at the animation fully objectively. Given enough time and resources these issues could be addressed by adding detail to the scene by professional artists and the system should be then evaluated by a truly random audience (not dominated by CS students, with a greater amount of participants).

The generated scenes reach an acceptable level of realism, however they are not realistic enough to deem the NLP method of generating animation to be better than other, traditional ones. 

\subsection{Preference}
Surprisingly, although the generated animations were not highly realistic, they were often preferred to the original ones. It seems that the recreated animation may have been over the top, while in the original scenes there was simply not much going on (the gestures being very subtle and ambiguous). It seems that the participants found the slightly overdone body language more enjoyable and less boring than the blandness and genericness of the originals.

Realism itself is often not a key aspect to a game's success. It might be a great advantage for a game to feature scenes that are similarly or slightly less realistic than the norm when those animations are more engaging. This might be particularly true when a given game is not trying to be realistic (it might be over the top and cartoony on purpose).


\subsection{Adjustments}
The results regarding adjustments of the animation were not surprising. The software was never intended to produce perfect animation which needs no adjustments before shipping (there is hardly any software capable of that). Moderate adjustments of the animation are perfectly acceptable. If this software was ever to become fully commercially viable, it would be a good idea to further develop the animation generation. Better camera work, smooth blending of the movements, etc. can be done automatically and would further minimize the need for manual adjustments.


\section{Other findings \label{sec:otherfindings}}
In previous sections I mentioned that the scenes chosen for the questionnaire are not \textit{fully} representative of all the dialogue scenes in a game. When a scene satisfies some conditions, the software clearly under-performs. Below are some of such cases that I discovered.

\subsection{Mismatch between animation database and the `feel' of the game}
The software produces undesirable results when there is a clear mismatch between the provided gestures and the `feel' of the game. By `feel' I mean the general topic and overtone of the game. For instance, a game such as any in the Mass Effect series has a very strict and militaristic feel. While characters still converse in a more casual environment, a certain level of professionalism is always required of them. A game such as `Yooka-Laylee' also contains a fair number of dialogue scenes, but the general feel of the game is childish and cartoon-like. The conversations happen in a much more quirky and exaggerated way.

I was not able to use the EBMD with the Mass Effect games. The EBMD provides very casual, relaxed gestures when almost every character in the Mass Effect game is military. While without context the scenes looked good, when keeping in mind that the scene happens in a militaristic environment one can immediately see that while the characters move and gesticulate in a natural manner, this is not appropriate in a current situation.

This however is not an inherent problem with the software, as it allows for using a custom animation database. It could potentially be even extended to support a different set of animations for each important character in order to underline their personality even more. However, because I used only EBMD for this project, I am unable to test how successful would the software be with scenes that have different contexts.

\subsection{Exposition dialogue}
In media such as books, films and games exposition is pretty common. Exposition is an aspect of narration that focuses on introducing character's backstories, prior plot events, context and other background information. Because typically games feature less narration than books or films, most of the exposition is handled through the dialogue.

The exposition dialogue is already often unnatural and seems out of place. The software generating those scenes made them seem even less natural. During exposition dialogue, the characters should remain rather neutral, as often they will be describing events that happened in the past, or that they have no direct connection with. The software however will find emotions in such dialogue and make characters overjoyed or fearing while there is nothing happening that would provoke those emotions. In many such dialogue scenes the software was simply over-sensitive.

This problem could potentially be solved if the exposition dialogue was tagged in the script by the writers. The emotions extracted from the text could be decreased by some factor to make them less intense. This of course assumes an increase in manual work required as the exposition dialogue would have to be manually tagged.


\subsection{Sarcasm and ambiguity}
NLP has struggled with sarcasm for a long time. IBM Watson Tone Analyzer seems to be no closer to solving this problem. The dialogues that contain sarcasm are simply misinterpreted and the final scenes look terribly out of place. For instance, a character saying `Oh, I'm so scared!' can be either scared, or indifferent and boasting about their courage. This might depend on the context and the tone of their voice. The software will however always associate this with fear and the resulting scene might use animations that convey opposite emotions to those that were meant to be conveyed.

By looking at many dialogue scenes, it became apparent that there is no way to overcome this problem using NLP only, as there are simply too many variables outside of the text that define possibility of sarcasm (context, setting, personality of the character, recent interactions between characters involved in the scene, shared history between conversing characters). The tone of voice might provide more information that would help solve this problem.

Similarly to the exposition problem, the sarcasm problem can also be solved by tagging it in the script. When using sarcasm the writers could indicate that and provide the intended emotion. That assumes a slight increase in the amount of work required and that all sarcasm present in the script is recognized and tagged.


\section{Summary}
The following is a breakdown of what the software managed and failed to achieve.

\noindent Successes:
\begin{itemize}
	\item The software is able to generate scenes quickly with minimal amount of manual work required.
	\item The animations do not need extensive manual polish before being used in a game (The software needs to provide smoother movements and blending between gestures).
	\item The software is flexible and capable of using custom character models and animation data.
	\item The software is cross-platform and the animations can be exported to a multitude of file formats.
	\item The generated scenes are often more enjoyable than the actual scenes in games.
\end{itemize}

\noindent Failures:
\begin{itemize}
	\item Generated scenes are not any more realistic than those featured in modern games.
	\item The software under-performs when dealing with sarcasm, exposition or ambiguity.
\end{itemize}

\noindent Other key points:
\begin{itemize}
	\item The lack of realism may be caused by the data provided by EBMD.
	\item The animation generation process can be improved further using methods already known to the industry.
	\item The NLP approach might in the future be combined with other methods (such as emotion analysis of the audio) to produce even better results.
\end{itemize}





\include{conclusion}

\appendix
\include{proof}

\bibliography{mybib}

\end{document}
