\documentclass[phd]{abdnthesis}

%% For citations, I would recommend natbib for its                          
%% flexibility, particularly when named citation styles are used, but                
%% it also has useful features for plain and those of that ilk.                      
%% The natbib package gives you the following definitons                             
%% that extend the simple \cite:                                                     
%   \citet{key} ==>>                Jones et al. (1990)                              
%   \citet*{key} ==>>               Jones, Baker, and Smith (1990)                   
%   \citep{key} ==>>                (Jones et al., 1990)                             
%   \citep*{key} ==>>               (Jones, Baker, and Smith, 1990)                  
%   \citep[chap. 2]{key} ==>>       (Jones et al., 1990, chap. 2)                    
%   \citep[e.g.][]{key} ==>>        (e.g. Jones et al., 1990)                        
%   \citep[e.g.][p. 32]{key} ==>>   (e.g. Jones et al., p. 32)                       
%   \citeauthor{key} ==>>           Jones et al.                                     
%   \citeauthor*{key} ==>>          Jones, Baker, and Smith                          
%   \citeyear{key} ==>>             1990                                             

\usepackage[numbers]{natbib}
\usepackage{hyperref}
\usepackage{scrextend}
\usepackage{pdfpages}
\usepackage{float}
\setlength{\bibsep}{0pt}
\bibliographystyle{apalike}

\usepackage[T1]{fontenc}
\usepackage{makecell}

\title{Using Natural Language Processing for Automatic Generation of Animated Dialogue Scenes in Video Games}
\author{Mikolaj \ Panasiuk}
% IMO this is a bit silly, but some like to include these. To remove,
% delete this declaration and remove the option from the
% \documentclass definition above.
%\qualifications{PhD, Computer Science, University College London, 1997\\%            
%BEng (Hons.) Electrical and Electronic Engineering, The University of Wales, Swansea, 1992}
\school{Department of Computing Science}

%%%% In the final submission of a thesis, this should only be the year
%%%% of submission.  However, it is useful to use \date{\today} for drafts so that
%%%% they don't get mixed up.
    
\date{2017}
\newcommand \videoshost{LINK TO VIDEOS}

%% It is useful to split the document up as chapters and include
%% them.  LaTeX will sort out all the numbering and cross-referencing
%% for you --- if you run it enough times!

%% If you want to include only a couple of chapters then use the
%% \includeonly{} command with a list of the file/chapter names that
%% you wish to include.  NB, this must be in the preamble.

\includeonly{introduction,background,analysis,faq, design, requirements, testingperf, usermanual, maintenance, evaluation, results, otherappend, conclusions, discussion}

\def\sfthing#1#2{\def#1{\mbox{{\small\normalfont\sffamily #2}}}}

\sfthing{\PP}{P}
\sfthing{\FF}{F}

%% This will make sure that all cross-references are correct (including
%% references to those file not included) but will produce a dvi
%% file with only those files/chapters you specify included.

\begin{document}

%%%% Create the title page and standard declaration.

\maketitle
\makedeclaration

%%%% Then the abstract and acknowledgements

\begin{abstract}
  Modern video games (especially those of the RPG genre) can feature over a hundred hours of dialogue. During the dialogue scene, the characters are expected to behave naturally and use body language in a way that is natural and corresponds to what they are saying. Due to the sheer amount of dialogue animation needed it is impossible to manually animate all the scenes. Many solutions to automatic generation of those scenes have been tried (most of which are not available publicly and still require a fair amount of manual work) with mixed results.
  
  This project aimed to develop a tool capable of generating 3D animated dialogue scenes with virtually none manual labour required. This is achieved by using natural language processing to extract the mood of characters directly from script and using a database of pre-recorded motion capture animations to assemble a dialogue scene. The tool was also aimed to be cross-platform, easy and cost-efficient to use, as such a tool would be most useful for small teams and amateur developers.
  
  The tool was evaluated by a real audience whose task were to compare scenes from actual games with scenes generated by the developed software. The tool succeeded in creating rather convincing animation that was often described as more enjoyable or engaging than the animations from games; Unfortunately the generated scenes lacked realism and the NLP approach was not accurate enough when dealing with sarcasm or ambiguity. Because of those issues the project did not achieve a full success.
  
  The software itself would need some serious improvements before becoming commercially viable. It is however developed enough to prove the usefulness of NLP in game animation. This approach has potential to be incorporated into a larger animation generation framework (where many approaches can collaborate to create the most lifelike animations) and even to be extended into other areas such as robotics. 
\end{abstract}

\begin{acknowledgements}
  I would like to thank Professor Ehud Reiter for his invaluable guidance, supervision and patience. 

  Secondly, I would like to thank the Blender Foundation for providing free, open-source, modern 3D content creation tools for over 15 years and the Max Planck Institute for Biological Cybernetics for making the Emotional Body Motion Database available publicly.
  
  I would also like to thank the BioWare Montreal studio for their failure in developing \textit{Mass Effect: Andromeda}. The spectacular fiasco of this game has brought attention to the importance of automatically generated dialogue animation and has been an inspiration for this project. I myself will keep pretending that this game has never happened and I am still happily waiting for the continuation of my favourite series (the same way I am still waiting for a sequel to \textit{The Matrix}). The following blank-verse poem commemorates the achievements, the failure and the disbandment of the BioWare Montreal studio:
  
  \bigskip
  \noindent \textbf{BioWare Montreal}\newline
  It seems like we hardly knew you,\newline
  probably because you only existed for 9 years and were very small\newline
  The more surprising is the fact that\newline
  you have been put in charge of the most significant of BioWare's franchises\newline
  \newline
  Pity,\newline
  because you looked like a pretty cool place to work at\newline
  Even though Glassdoor reviewers complained that there are no windows at the office\newline
  and that they all feel like nocturnal creatures.\newline
  Montreal also seems like a good place to live I guess\newline
  despite all the snow in the winter and the roadwork in the summer\newline
  But how would I know, I have never been to Canada anyway\newline
  \newline
  You should have went gently into that good night\newline
  But you didn't\newline
  And now your game is a laughing stock for the entire gaming community\newline
  Press F to pay respects\newline
  F
  
  
  
  
\end{acknowledgements}


% Tables 
\tableofcontents
\listoffigures
\listoftables

% Body
\chapter{Introduction\label{chap:introduction}}

The game industry now is bigger than ever before and still growing. Along with technological advancements as well as a rise in popularity games themselves have become bigger and more polished. With increasing size and quality the number of man-hours required to build a game rises drastically. A lot of work is being put into creating tools that enable faster creation of content. However, there is still a lot left to be optimized and automated.

One domain of game development that drains lots of man-hours into monotonous processes that could potentially be automated is animation. Most studios will rarely animate everything by hand as there is simply too much content to cover. While some animated content needs to be very polished (cutscenes - action sequences, parts of game that greatly influence the plot development), some animation might be cruder (dialogue sequences). Games like the RPGs\footnote{RPG - Role-playing game.} will feature a lot of dialogue - during the dialogue the characters cannot just stand still as it would negatively impact player's immersion in the game world. The characters must move and perform gestures that underline their speech in a natural manner. These animations cannot be all done by hand because of their sheer cumulative length. For instance:

\begin{itemize}
	\item Mass Effect Andromeda and Fallout:  New Vegas feature 65,000 lines of dialogue\footnote{\url{https://www.pcgamer.com/mass-effect-andromeda-has-over-1200-speaking-characters/}}\footnote{\url{https://www.pcinvasion.com/fallout-new-vegas-will-have-65000-lines-of-dialogue}}.
	
	\item The Witcher 3 features roughly 35 hours of dialogue scenes\footnote{\url{https://www.pcgamer.com/most-of-the-witcher-3s-dialogue-scenes-was-animated-by-an-algorithm/}}.
	
\end{itemize}

The main challenge is to make the dialogue scenes (and other automatically generated animation) look indistinguishable from the cutscenes (usually manually created and well-polished). In many games a player will experience watching well-polished animation that immediately switches to poor, clunky and unrealistic animation. The lesser in quality dialogue scenes break the immersion of the player and negatively impact the overall experience. Easier, faster and better quality methods of generating dialogue scenes would be a great asset to the gaming industry.

\section{Motivation}

The purpose of this project is to develop a tool that helps generate animated dialogue scenes and minimizes the amount of manual work by using natural language processing. Generating the scenes directly from script would pose several benefits:

\begin{itemize}
	\item A script is often written to design the plot of a game. The same script could be fed into a program to generate the animations.
	\item The script is semi-structured natural language. Using natural language would help shorten the gap between artists, writers, animators and technicians.
	\item Such tool can be used to prototype scenes quickly and easily.
	\item Such tool program can be used by people who know nothing about animation.
\end{itemize}

The program I propose would create prototype animation with almost no amount of manual work required. Requiring less manual labour than other approaches would make this tool a very cost efficient way to create realistic dialogue scenes.

\section{Objectives}

The Projects Objectives are as follows:

\noindent {\bf Develop a tool able to interpret a natural language script}

\noindent The tool must be able to read a semi-structured script and recognize dialogue lines and emotions of the characters.

\noindent {\bf Develop a tool able to blend a final dialogue scene}

\noindent The tool must be able to output a fully editable dialogue scene. The scene is assembled using pre-made motion capture clips.

\bigskip
The scenes created by the software will be very crude and unpolished. Scenes generated by the tool will need to be polished manually - the amount of polish required can be decided by assessing the importance of a given scene. However there is a chance that the scene quality will be much inferior to scenes generated by similar tools that use different approaches. Therefore an important question this project tries to answer is whether taking the NLP approach to animation is feasible in the games industry given current technology.

\chapter{Background and Related Work \label{chap:background}}

ur mom gay
\chapter{Analysis \label{chap:analysis}}

\noindent The project  methodology,  development  tools  and  technologies  and  project  risk assessment are defined in this following chapter.


\section{Methodology}

The activities required in order to develop the project are listed below:

\begin{itemize}
\item Develop a project plan.
\item Review literature. Review existing software, learn about other approaches, analyze existing research.
\item Find and assemble a set of relevant motion capture animated clips.
\item Prototype a module that uses processes the script and outputs a structured file that can be used to generate the animated scene.
\item Tag and classify animation clips. Categorize clips by emotion, length, etc.
\item Prototype a Blender extension that uses the created structured files to output an animated scene.
\item Iterate on the existing software adding new features.
\item Evaluate the prototype with real audience.
\end{itemize}

The generated animation must be evaluated with help of a real audience. How animation is perceived is subjective and cannot be decided by one person only. The evaluators will be asked to watch animated dialogue clips from various games and then watch this dialogue recreated using the proposed software. The evaluators will be asked to fill an evaluation survey. This way it will be possible to determined the usefulness and succesfulness of the proposed software.


\section{Technologies and Resources}

\begin{enumerate}
\item Emotion analysis - The first module of the project focuses on NLP. This module should be extract emotions and actions from the script. The most important tool used will be IBM Watson Tone Analyzer. If that tool turns out to be for any reason ineffective or imperfect, a customary tool (naive bayes classifier or a keyword classifier) can be built for that task. Actions can be extracted using a variety of information extraction software such as MITIE or Ollie.

\item Motion capture - The EMBD (emotional body motion database) will be used to source the animations of body movement and gestures. The database provides the recordings as BVH files which is convenient as most animation software (such as Blender or Maya) can import BVH files. The emotional metadata about the animated clips will be stored in an SQLite database. Storing the data using this method will allow the data to be easily and quickly searchable while less complicated than using a full fledged database software (such as MySQL od PostgreSQL) (very few tables are needed, there is no need to use advanced DB software).

\item Animation software - As aforementioned in section ~\ref{sec:animchoice}, the animation software that satisfies all the specified requirements is Blender. Blender supports all the necessary modelling and animation features. It also supports creating extensions allowing the animation to be created and assembled by code.

\item Programming - Python 3 will be used to implement the software. Python is the only language supported for add-on development for Blender. For other modules that do not rely on Blender, Python was chosen due to its development speed and in order to keep consistency among modules.

\end{enumerate}

\section{Risk Analysis}
\label{sec:riskanal}

\begin{table}[!ht]
	\centering
	\small
	
	\begin{tabular}{ |p{11em} |p{23.8em}|p{4em}| }
	 \hline
		\thead{Risk} & \thead{Mitigation} & \thead{Level} \\
	 \hline
	 	Time delays caused by workload/illness & Follow the project plan closely. Focus on developing a minimum working prototype first. & High \\
	 \hline
		Natural Language Approaches are too inaccurate & Use more structured software. Try to find other approaches to this problem. If there seems to be no solution, I can use the findings and existing software to argue that the technology has not yet reached a level that would make this project viable.  &  Low \\
	\hline
		Motions obtained from the EBMD are not expressive and unambiguous enough to create realistic and convincing scenes & Even with bad quality animations it should be possible to determine whether this project has future potential, given a better motion capture database is used. & Moderate \\
	\hline
	\end{tabular}

	 \caption{Risk Assessment}
	 \label{tab:riskassessment}
	 
\end{table}
\chapter{Requirements Specification\label{chap:requirements}}

\noindent This section describes functional and non-functional requirements of the proposed software.


\section{Functional Requirements}

\noindent {\bf FR1 Analysis of a semi structure script}

\noindent The software takes a semi structured script as input. The structure of a script must resemble a structure of a movie script. The script provides 3 kinds of information - characters involved in a scene, dialogue lines and actions performed by characters during the scenes. The software must be able to extract actions and emotions from the script.

\bigskip

\noindent {\bf FR2 Find relevant animation clip}

\noindent The software must take information about character's emotions and actions and choose animation clip that best represent's characters behaviour.

\bigskip

\noindent {\bf FR3 Assemble final scene}

\noindent The software takes information generated by other modules and outputs an animated scene. The outputted scene must be fully editable, enabling various adjustments before rendering.


\section{Non-Functional Requirements}

\begin{itemize}
\item The user should be able to use the software with a custom motion-capture database (Usability).
\item The user should be able to assign differenct character models to different characters (Usability).
\item The user should be able to fully customize and edit the final scene (Usability).
\item The software is designed to automatically create big amounts of animated scenes. The time of generating a scene is not a high priority, but it must be reasonable (Peformance).
\item The software must support common animation file formats (fbx, bvh) (Portability).
\end{itemize}

\chapter{Design And Architecture\label{chap:design}}

This chapter describes the design of the system. This includes both the underlying decisions of the system as well as the user interface design. The chapter also presents on a more detailed view of the system's architecture.

\section{System Design}

The system has to work in two major steps. The first step is to take semi-structured natural language script and analyze it. This means that the system must analyze each dialogue line and infer some emotional values from them. Using those emotional values and the length of the dialogue line, the system must find best matching animation clips from the motion capture database. The results of this step is a file that specifies which characters say which lines of dialogue and it also specifies which animated clips accompany those dialogue lines.

After this step is completed, the file must be intepreted into a final scene. The importer must be able to read the file, import required models and animations and generate the final scene. As every animation software and game engine is a little different, each of them would need a custom importer. Those would be very similar in principle, but differ slightly because of the implementation of given software. For my project I have created the importer for Blender. This is because Blender is open source and available to anyone, as well as I am the most familiar with this software.


\section{Emotions}

For my project I have decided to only use the following emotions: joy, fear, disgust, anger and sadness. Tradinionally, surprise is also part of the six basic emotions model by Paul Ekman and Wallace V. Friesen. However, IBM Watson Tone Analyzer, that I am using for sentiment analysis, is unable to detect surprise in the text (more on that in section ~\ref{sec:emoanal}) - therefore I had to abstain from using this emotion in this project. The emotions are expressed as decimal numbers between 0 and 1 - this identifies whether a given emotion is present in a specific motion clip or text and how intense that emotion is. It also allows to create a mixture of emotions in order to represent more complex emotions.

\section{Architecture}

The architecture of the system is essentially a pipeline. The modules process resources and pass them onto the next module. They can be completely unaware of each other. This allows for a lot of flexibility and helps achieve some of the requirements. The emotion analysis API can be replaced with a different one, the user can use a custom motion capture database, and a different importer can be created if the user wishes to use software other than blender.


There are three main modules:
\begin{itemize}
\item Text Analysis
\item Animation Clip Matcher
\item Animation Generator
\end{itemize}

The Text Analysis module parses the text and analyses emotion using some API. It takes a script as an input and passess the parsed and analyzed text to the Animation Clip Matcher.

The Animation Clip Matcher uses the motion capture database to find the best animation clips to accompany the speech. The Animation Clip Matcher must take into account the emotion analysis of the text and find animations with similar emotional score. Another important constraint is the time of the animation. The animated clips have different lengths and a chosen clip must not be significantly longer than the spoken/read text. In case the animated clip is shorter, the Matcher must choose more than one matching clip. The Matcher outputs a JSON file which specifies all dialogue lines; it states which characters perform which emotional gesture actions and where is the animation file located.

The Animation Generator reads the JSON file and imports all needed animations. The user is now able to assign character model for each character and run the generator. The generator will assemble the animations in correct order, focus the camera on a currently speaking character and add subtitles. The output is an animated scene that can be either manually edited, exported to a file or rendered.

The full diagram of the system can be seen in figure ~\ref{fig:architecture}.

\begin{figure}[!ht]
\centerline{\fbox{\includegraphics[width = 30em]{img/architecture.png}}}
\caption{The pipeline architecture of the system}\label{fig:architecture}
\end{figure}

\section{Character Armature and Model}

TO DO

\section{EMBD Iimporter}

\section{Input}

The input of the system is as follows:

The input is a file that represents a dialogue. Each character that participates in the scene must ble clearly stated. Each dialogue line must have a dialog line must have a character clearly associated with it. Character names are specified after five tabs. Dialogue Lines are specified after three tabs. Each file must end with "ENDSCRIPT" with no indentation.

I used this format as this format is often used to represent movie scripts. One can find hundreds of scripts saved in this format on The Internet Movie Script Database (\url{www.imsdb.com}). The example input can be seen in figure ~\ref{fig:inputscript}.


\begin{figure}[!ht]
\centerline{\fbox{\includegraphics[width = 30em]{img/script.png}}}
\caption{An example of system's input}\label{fig:inputscript}
\end{figure}


\section{Emotion Analysis} 
\label{sec:emoanal}
As aforementioned, there are many ways to perform emotion analysis of the text. I have chosen to use IBM Watson Tone analyzer for this task as it offers a state-of-the-art service accessible for prototyping (a lot of tools offering high quality emotional analysis are bult for commercial purposes and not available without paying high fees).

The system takes each dialogue line and sends to IBM Watson Tone Analyzer for analysis. Watson's REST API is used to accomplish that. Watson returns all the emotions with values between 0 and 1 found in the text. Each dialogue line now has emotional values assigned to it.

\section{Database}

The motion capture database consists of two main parts: the files that contain the animated clips and a database that holds the metadata about the animations.

Most animation come from the Max Planck's Research Institute Emotional Body Motion Database. This database is a set of motion capture clips performed by actors. The clips are supposed to represent emotions in various settings. Most of these animations need minor changes in order to be compatible with my system (e.g. a lot of clips were recorded while the actor was sitting, so the leg keyframes must be removed to make the model stand). The animations are exported to FBX files (proprietary file format owned by Autodesk, widely supported by many different frameworks) one animated clip per file.

The database is implemented using SQLite. It is the perfect tool for this task as the database needs to be simple, relatively small and easily searchable. The animation metadata is held in a table that look like this FIGURE. The emotional values of each animated clip need to be manually adjusted. When all values are set to zero, it means that the clip carries no emotional impact (neutral). An example of a few database records can be seen in figure ~\ref{fig:db}.

\begin{figure}[!ht]
\centerline{\fbox{\includegraphics[width = 30em]{img/db.png}}}
\caption{Database of animation clips}\label{fig:db}
\end{figure}

The other purpose of the database is to store the information about the models. The important information about a model is its name, file location, camera offset and rotation. Because models may be of different shapes and sizes, camera positioning during dialogue will not be the same. The database allows to specify a desirable camera position relative to the character that will allow to fully capture the character at a good angle. Example of a few model database records can be seen in figure ~\ref{fig:dbmodel}.

\begin{figure}[!ht]
\centerline{\fbox{\includegraphics[width = 30em]{img/dbmodel.png}}}
\caption{Database of character models}\label{fig:dbmodel}
\end{figure}


\section{Matcher}

TO DO


\section{Animation Generator (Importer)}

As a forementioned, in this project I have used Blender to support the animation generator model. The generator is a Blender addon.  It can either be used as a script, or be installed as an addon to be fully and permanently available within Blender.

The generator first reads the JSON file prepared by previous modules and loads required animation clips from files. The user needs assign a 3D character model to each character participating in the dialogue. The required 3D character models are animated and the camera is positioned accordingly to the model's metadata stored in the database. The animated characters are positioned to directly face each other (the program currently supports only up to two characters in one scene). Each animated action is assigned to a corresponding character model and then pushed on a separate NLA strip (figure ~\ref{fig:nla}). The program keeps track of how many frames are being used so that newly added actions begin just when the previous actions have ended.

To setup some basic background and feel, a floor (plane) is added beneath the characters and two directional lights cast light down from above the characters.

The camera is added to the scene and positioned using the metadata retrievec from the database. At the start of each character's actions the camera is positioned and rotated to focus at the character performing an action.

Subtitles are added as Video Editor Sequence Strips spanning between frames that correspond to the animation (figure ~\ref{fig:substrips}).

\begin{figure}[!ht]
\centerline{\fbox{\includegraphics[width = 30em]{img/nla.png}}}
\caption{NLA strips of the final animation}\label{fig:nla}
\end{figure}
\begin{figure}[!ht]
\centerline{\fbox{\includegraphics[width = 30em]{img/substrips.png}}}
\caption{Video strips featuring subtitles}\label{fig:substrips}
\end{figure}

The final animation can be edited and adjusted in Blender (figure ~\ref{fig:finalblend}), rendered (figure ~\ref{fig:finalrend}) or exported to file.


\begin{figure}[!ht]
\centerline{\fbox{\includegraphics[width = 30em]{img/finalblend.png}}}
\caption{Editing the final animation in Blender}\label{fig:finalblend}
\end{figure}
\begin{figure}[!ht]
\centerline{\fbox{\includegraphics[width = 30em]{img/finalrend.png}}}
\caption{Rendered final animation}\label{fig:finalrend}
\end{figure}

\section{User Interface}

The Animation Generator module is the only module that features a Graphical UI. The module is embedded into Blender and uses extends Blender's interface. 

The UI is minimalistic and simple to use. Upon installation and activation a new tab is added to the menu on the left hand side. Figure ~\ref {fig:ui_main}. Rectangle 1 represents the left hand side menu. Number 2 shows the tabs where the bottommost one belongs to the extension. Number 3 shows the actual UI of the extension.

\begin{figure}[!ht]
\centerline{\includegraphics[width = 18em]{img/ui_main.png}}
\caption{The addon UI}\label{fig:ui_main}
\end{figure}

From here the user is able to either clean the scene (might be useful especially if something unexpected happens or the work of the extension is interrupted) or prepare the scene. To prepare the scene the user must first initialize the four variables. The user needs to point the program to where the animations and models are stored as well as the database file and the scene file (JSON file generated by previous modules).

Upon pressing `Prepare` the program will prepare data for generating a scene. Required animations will be imported and the scene file will be parsed. There is one more step left to finalize the animation. When preparation is finished, a new menu will pop up below the currently existing one. The menu can be seen in figure ~\ref{fig:ui_finalize}. In the menu the user is required to assign a character model to each character existing in the scene. Upon pressing `Finalize` the full scene will be generated complete with lighting, camera movement and subtitles.

\begin{figure}[!ht]
\centerline{\includegraphics[width = 18em]{img/ui_finalize.png}}
\caption{The finalization step (new menu highlighted by white rectangle)}\label{fig:ui_finalize}
\end{figure}

They key aspect of the interface is its simplicity as it allows users to generate relatively complicated animation sequences by essentially using two buttons, with no knowledge about animation and little Blender expertise required. The UI also provides flexibility as the user can decide upon which character models to use.






\chapter{Testing and Performance \label{chap:testingperf}}

This section describes testing and potential issues of the developed software.

\section{Animation Generator}
There are some serious limitations to testing of this module. It is hard for a computer to decide whether the animation looks natural. Some animations imported from EMDB experience import problems which results in an animation with broken and unnatural movements. While manually analysing the emotional values of the animations, it was important to check each animations for movements that may have been incorrectly imported. Such animations cannot be used with the software.

The animation generator is now able to handle only up to two characters in the scene (although the module was designed with extensibility in mind). If a scene JSON file contains information on more than two characters (or no characters at all) the scene will not be assembled and an error message will be displayed instead. In case the scene file is corrupted and cannot be parsed the user will also be presented with an error message.

\section{Performance}
One of the important requirements of the software was the speed of execution. For creating an animation consisting of 6 dialogue lines using the first module takes up to 18 seconds. The matcher module is so quick that it's execution does not influence the total execution time. The generator module takes another four seconds on average. This results in an execution speed of about 3.6 seconds per dialogue line. This fulfils the requirement as the time required to assemble an animation is incomparably less than doing that manually and is not a limitation for an animator working on the scene.
\chapter{Evaluation \label{chap:eval}}
This chapter describes how the system was evaluated and explains why such a method of evaluation was chosen. The evaluation aims to determine whether the created system was successful or not and due to what reasons. 

\section{Problems with evaluation}
Evaluation of animation is quite problematic in nature. The main issue is that how an animation is perceived is very subjective - it is near impossible to computationally determine whether an animation looks natural, whether an animation goes in pair with the speech and if the whole `feel` of the dialogue scene is good. The animations are intended to be received by human audience and it the human factor is essential to determining whether the developed system is successful. I myself however am not fit to decide that - I have spent far too much time creating the animated scenes and watching both failed and successful results. Because of that I am obviously seriously biased and cannot judge the animations from an objective standpoint.


\section{Survey}
In order to account for my bias, I have decided to use interview people about the generated animations. People who have not had a part in the development of the software can provide valuable opinions.


\subsection{Methodology}
People may find it hard to provide opinion about quality of an animation if they have nothing to base their answer upon (especially if they are not animators or developers). There needs to be some sort of a benchmark or a control sample to enable people to judge the animations more efficiently.

To account for that I have chosen the following approach: I would first find a dialogue scene in a relatively popular, successful game and recreate it using the developed software. The participants would be shown both the original and recreated scene and describe how they compare - they would be asked to decide whether using the software has improved or decreased the quality of the scene. 

\medskip
\subsubsection{Preparation}
I have chosen three games from which I extracted the dialogue scenes. I did not want to extract a few dialogue scenes from one game in order to provide variety, but I also did not want to use more than three games as it would make the questionnaire too long and make the participant loose interest or become confused.

The three games are all of the RPG genre and differ mostly by the year of release - I have used a relatively old game, relatively new game and a state-of-the-art game - for if the software is deemed to produce animation of lower quality than the state-of-the-art game, comparing it to an old game may help decide whether the software simply needs improvements or if it the approach itself is wrong.

Not simply any game could have been chosen. In my project I have to rely on EMBD to generate the animations - and the motion capture data provided by the database is not suitable for any game. For example, if a game features a lot of soldier to soldier military interactions it is expected that the characters will behave in a more serious, less expressive manner (I evaluate on that in section ~\ref{sec:otherfindings}). The EMBD does not supply such animations that would be useful in this situation and using a wrong game would cause unnecessary confusion of the participants (for such games a custom motion database can be used with the software). The chosen games must feature some of more relaxed, natural conversations. This certainly tightens the pool of games usable for the evaluation, but I believe that the chosen games are a rather representative sample of the RPG genre.

The chosen games are:
\begin{itemize}
	\item Fallout 3 (released in 2008)
	\item Fallout 4 (released in 2015)
	\item Horizon: Zero Dawn (released in 2017)
\end{itemize}

The dialogues from the games were chosen randomly, however there were some constraints that limited the choice:
\begin{itemize}
	\item The scene cannot feature more than two characters (The software was designed with extendibility in mind, however at the moment it does not support more than two characters).
	\item The dialogue has to convey some emotions (dialogues which result in only neutral gestures are not truly representative of the software).
	\item The dialogue cannot be exposition dialogue (Why that is the case and what exactly is `exposition` dialogue I explain in sections ~\ref{sec:evalotherfindings} and ~\ref{sec:otherfindings}).
\end{itemize}

Before conducting an interview it is important to know whether the interviewee has experience with games and animations and development. The answers of people knowledgeable in this area might differ from those provided by people new to this area and it might be helpful to be able to distinguish between them.

\medskip
\subsubsection{The questionnaire}
The questionnaire consists of four sections. The first section provides a little detail about what the software is trying to achieve and what is expected of the participant. This section also asks whether the participant considers themselves to be knowledgeable about games or animations. This is the only question in that might be viewed as personal.

The other sections all ask the same set of questions about different pairs of videos. At the start of each section the participant will be shown two videos - an original and a recreated dialogue scene. After watching both of them they will be asked a couple of questions that help determine whether they find the videos realistic, whether the animations correspond to the conveyed gestures and whether they think that using the software improves the scenes. The participants are able to evaluate in more detail on the answers if they wish so. The questions that are designed to be answered on a linear scale resemble the Likert scale\footnote{The Likert scale is a psychometric scale created by a psychologist Rensis Likert. It is commonly used in questionnaires in order to determine whether the participants agree or disagree with given statements.}.

The surveys were conducted as an one-on-one interview if possible. If the participants were unavailable or preferred not to do this personally an online version of the questionnaire was provided. The participation in the questionnaire was fully voluntary. The participants were presented with a consent form before answering the questions and were provided with the details of the software and how the study is going to be conducted. No aspect of the software or the study was hidden from the participants and they were able to both request more details or quit any time they wished to do so.

As mentioned before, the preferred approach was a supervised survey as it is important that the participants correctly understand what the project is about and what it is trying to achieve. The survey was targeted mostly at my colleagues (Bachelor computing science students), as certain level of computing science and application development expertise is desirable. That is because people who are unacquainted with prototyping and development might focus too much on how unpolished the animation is (since it is only a prototype and the animations are rather crude - there is no background in the scene, no audio and no texturing/graphics) than on the gestures and dialogue itself.

The full questionnaire can be found in the appendices in section ~\ref{sec:appquestionnaire}.



\section{Other findings \label{sec:evalotherfindings}}
Apart from the findings of the questionnaire I also wish to describe my own realisations that I devised during the development of the software. I know that I cannot use them to decide the successfulness of the software because of my bias towards it, however I think there are a few aspects worth mentioning that will not be found by the questionnaire participants.

As I was testing the software many times and I tried to recreate many scenes from RPG games I have noticed that the software does not perform as well if certain conditions are not satisfied. For example: the general `feel` of the game must match the gestures provided by the motion database, the software does not perform with scenes that contain exposition or sarcasm and more. I will explain those issues in more detail in section ~\ref{sec:otherfindings}.





\chapter{Results \label{chap:results}}

\section{Survey}
The survey was completed by 12 people in total. Half of the participants have described themselves as well-acquainted with the area of games or animation. One person was not sure and the rest was not acquainted with that topic. However it appears that this is never strongly correlated (using Pearson correlation coefficient) to any of the other answers. It seems that having experience in games or animation does not significantly influence how the participants perceived the animations.

The participants were asked to decide whether the animations presented to them were realistic. The question was asked about each of the six scenes. The t-test was performed taking into account the answers about all the original scenes and the recreated scenes. The p-value of 0.7614 shows weak statistical significance of the result. The realism of the scenes was described on a one to five scale\footnote{The participants were asked to say whether they agree with the statement  `video X is realistic'. The answer was taken on a one to five scale where one means `strongly agree' and five means `strongly disagree'}, where the average score for both original and recreated scenes is about 3.1. These results show that animations generated by this project are no more realistic than the originals and that both original and recreated scenes are hardly realistic at all.
	
A similar set of questions was asked about the effectiveness in conveying emotions. The participants were asked to decide whether the animations correctly convey the feelings and emotions of the characters in the scene. This time the results yielded the p-value of 0.4657 showing a higher significance of the result. The participants' answers suggest that the animations reproduced by the software are slightly better at conveying feelings and emotions (the mean score of the recreated animations was by 0.3 (out of 5) better than the originals). These results show a very slight advantage of the recreated animations over the original ones.

The figure ~\ref{fig:realism_graph} shows a breakdown of these results for each game. The graph shows that the software is capable of creating more realistic animation than animation featured in an old game, but struggles to keep up with newer titles. The recreated animations are on average considered more realistic than animation from a 2008 game. The software produces animations of similar realism as animations featured in a 2015 game and is slightly outperformed by a 2017 game. One thing that is worth noticing is that the data suggests that the recreated scenes are capable of conveying the emotions and feelings correctly, however this is not enough for them to be seen as `realistic'.

\begin{figure}[!ht]
	\centerline{\includegraphics[width = 42em]{img/results/realism.png}}
	\caption{Average perceived realism of the animations}\label{fig:realism_graph}
\end{figure}



Despite the not so positive results above, the participants actually seem to prefer the recreated videos. When asked if the recreated animation makes the scene more enjoyable almost 70\% of participants agreed (figure ~\ref{fig:improves_graph}). Similarly, over 60\% of participants (figure ~\ref{fig:prefer_graph}) pointed to the recreated animation when asked which animation do they subjectively prefer. The interviewees justified their choice by statements such as `Body language matches with what is being said' or `More accurate expression of feelings'. Adversely, the supporters of the original animations described the recreated animation as `over the top' and with `weird hand movements'. The answers do not differ greatly game to game.

\begin{figure}[!ht]
	\centerline{\includegraphics[width = 42em]{img/results/improves.png}}
	\caption{Does the recreated animation make the dialogue scene more enjoyable or engaging?}\label{fig:improves_graph}
\end{figure}

\begin{figure}[!ht]
\centerline{\includegraphics[width = 42em]{img/results/prefer.png}}
\caption{Which animation do you personally prefer?}\label{fig:prefer_graph}
\end{figure}


The last important piece of information gathered from the participants was whether they though that the animations generated by the software need adjustments before being used in a finished game. The answers lean slightly into the `needs serious adjustments' side, however the average suggests that the animations need a moderate amount of adjustments.

\begin{figure}[!ht]
	\centerline{\includegraphics[width = 42em]{img/results/adjustments.png}}
	\caption{Do you think that the recreated animation needs adjustments (before being used in a finished game)?}\label{fig:adjustments_graph}
\end{figure}












\chapter{Discussion \label{chap:discussion}}
This section discusses and evaluates the findings accumulated during the development and evaluation of the software.

\section{Survey results}

\subsection{Realism}
According to the results of the survey the realism of the generated animations is rather disappointing. The animations managed however to be relatively as realistic as an animation from a quite recent game, meaning they might be good enough to be used in a game (given how little it takes to generate them). The very important aspect is that the generated animation were not seen as realistic, but were seen as correct (matching speech with emotions and body language). It is interesting that people often preferred the recreated scenes even when they were less realistic than the originals. Interviewees dissatisfied with the recreated animations said that the movements of characters are `weird' and that the software is too sensitive, with characters overreacting to the actual situation. 

The survey successfully carried out its first objective, showing that the NLP approach did not produce animations more (or much less) realistic than the originals. There are a few explanations to that. The possible causes have been described in section ~\ref{sec:riskanal}.

The lack of realism was probably not caused by the emotion analysis. The interviewees rather agreed that the emotions in the speech match the body language. The results suggest that the problems lies within the movements itself. 

Both the motion capture data quality and the uncanny valley principle are possible causes of those results. It is impossible to say now which one is more important. A similar study which uses a different motion database could be helpful in determining that.

It is also possible that the results are skewed because of the crudeness of the entire scene. The models are rather bulky and poorly detailed, untextured, there is no background or audio. That is because the software focuses solely on the animation. However, the interviewees are not used to watching animation being developed and might instinctively see such a scene as less realistic and be unable to look at the animation fully objectively. Given enough time and resources these issues could be addressed by adding detail to the scene by professional artists and the system should be then evaluated by a truly random audience (not dominated by CS students, with a greater amount of participants).

The generated scenes reach an acceptable level of realism, however they are not realistic enough to deem the NLP method of generating animation to be better than other, traditional ones. 

\subsection{Preference}
Surprisingly, although the generated animations were not highly realistic, they were often preferred to the original ones. It seems that the recreated animation may have been over the top, while in the original scenes there was simply not much going on (the gestures being very subtle and ambiguous). It seems that the participants found the slightly overdone body language more enjoyable and less boring than the blandness and genericness of the originals.

Realism itself is often not a key aspect to a game's success. It might be a great advantage for a game to feature scenes that are similarly or slightly less realistic than the norm when those animations are more engaging. This might be particularly true when a given game is not trying to be realistic (it might be over the top and cartoony on purpose).


\subsection{Adjustments}
The results regarding adjustments of the animation were not surprising. The software was never intended to produce perfect animation which needs no adjustments before shipping (there is hardly any software capable of that). Moderate adjustments of the animation are perfectly acceptable. If this software was ever to become fully commercially viable, it would be a good idea to further develop the animation generation. Better camera work, smooth blending of the movements, etc. can be done automatically and would further minimize the need for manual adjustments.


\section{Other findings \label{sec:otherfindings}}
In previous sections I mentioned that the scenes chosen for the questionnaire are not \textit{fully} representative of all the dialogue scenes in a game. When a scene satisfies some conditions, the software clearly under-performs. Below are some of such cases that I discovered.

\subsection{Mismatch between animation database and the `feel' of the game}
The software produces undesirable results when there is a clear mismatch between the provided gestures and the `feel' of the game. By `feel' I mean the general topic and overtone of the game. For instance, a game such as any in the Mass Effect series has a very strict and militaristic feel. While characters still converse in a more casual environment, a certain level of professionalism is always required of them. A game such as `Yooka-Laylee' also contains a fair number of dialogue scenes, but the general feel of the game is childish and cartoon-like. The conversations happen in a much more quirky and exaggerated way.

I was not able to use the EBMD with the Mass Effect games. The EBMD provides very casual, relaxed gestures when almost every character in the Mass Effect game is military. While without context the scenes looked good, when keeping in mind that the scene happens in a militaristic environment one can immediately see that while the characters move and gesticulate in a natural manner, this is not appropriate in a current situation.

This however is not an inherent problem with the software, as it allows for using a custom animation database. It could potentially be even extended to support a different set of animations for each important character in order to underline their personality even more. However, because I used only EBMD for this project, I am unable to test how successful would the software be with scenes that have different contexts.

\subsection{Exposition dialogue}
In media such as books, films and games exposition is pretty common. Exposition is an aspect of narration that focuses on introducing character's backstories, prior plot events, context and other background information. Because typically games feature less narration than books or films, most of the exposition is handled through the dialogue.

The exposition dialogue is already often unnatural and seems out of place. The software generating those scenes made them seem even less natural. During exposition dialogue, the characters should remain rather neutral, as often they will be describing events that happened in the past, or that they have no direct connection with. The software however will find emotions in such dialogue and make characters overjoyed or fearing while there is nothing happening that would provoke those emotions. In many such dialogue scenes the software was simply over-sensitive.

This problem could potentially be solved if the exposition dialogue was tagged in the script by the writers. The emotions extracted from the text could be decreased by some factor to make them less intense. This of course assumes an increase in manual work required as the exposition dialogue would have to be manually tagged.


\subsection{Sarcasm and ambiguity}
NLP has struggled with sarcasm for a long time. IBM Watson Tone Analyzer seems to be no closer to solving this problem. The dialogues that contain sarcasm are simply misinterpreted and the final scenes look terribly out of place. For instance, a character saying `Oh, I'm so scared!' can be either scared, or indifferent and boasting about their courage. This might depend on the context and the tone of their voice. The software will however always associate this with fear and the resulting scene might use animations that convey opposite emotions to those that were meant to be conveyed.

By looking at many dialogue scenes, it became apparent that there is no way to overcome this problem using NLP only, as there are simply too many variables outside of the text that define possibility of sarcasm (context, setting, personality of the character, recent interactions between characters involved in the scene, shared history between conversing characters). The tone of voice might provide more information that would help solve this problem.

Similarly to the exposition problem, the sarcasm problem can also be solved by tagging it in the script. When using sarcasm the writers could indicate that and provide the intended emotion. That assumes a slight increase in the amount of work required and that all sarcasm present in the script is recognized and tagged.


\section{Summary}
The following is a breakdown of what the software managed and failed to achieve.

\noindent Successes:
\begin{itemize}
	\item The software is able to generate scenes quickly with minimal amount of manual work required.
	\item The animations do not need extensive manual polish before being used in a game (The software needs to provide smoother movements and blending between gestures).
	\item The software is flexible and capable of using custom character models and animation data.
	\item The software is cross-platform and the animations can be exported to a multitude of file formats.
	\item The generated scenes are often more enjoyable than the actual scenes in games.
\end{itemize}

\noindent Failures:
\begin{itemize}
	\item Generated scenes are not any more realistic than those featured in modern games.
	\item The software under-performs when dealing with sarcasm, exposition or ambiguity.
\end{itemize}

\noindent Other key points:
\begin{itemize}
	\item The lack of realism may be caused by the data provided by EBMD.
	\item The animation generation process can be improved further using methods already known to the industry.
	\item The NLP approach might in the future be combined with other methods (such as emotion analysis of the audio) to produce even better results.
\end{itemize}





\chapter{Conclusions \label{chap:conclusion}}
conclhere

% Bib
\bibliography{mybib}

% Appendix
\appendix
\chapter{User Manual \label{chap:usermanual}}
The system is designed to be cross platform, however it is advised to use Windows as it is the only system on which the project was thoroughly tested.

\section{Requirements and installation \label{sec:requirements}}
Before using the software, several requirements must be fulfilled.

\subsection{Software and Libraries}
\subsubsection{Required software}
\begin{itemize}
	\item Python 2.7
	\item Pip
	\item Blender version 2.79 or newer
\end{itemize}

\subsubsection{Libraries}
Only for Linux systems: Download and install library python-dev (or python-devel)

Following python libraries must be installed:
\begin{itemize}
	\item setuptools
	\item watson-developer-cloud
	\item wget
\end{itemize}
\noindent The packages can be installed in bulk by running:

\indent `path\_to\_python/Scripts/pip install -r requirements`

\noindent in the main directory. As Blender usually comes with its own Python environment, you will need to run the command:

\indent `path\_to\_blender/version/python/Scripts/pip install -r requirements`

\noindent To install the libraries for Blender.

\subsection{Install Generator as Blender Addon \label{sec:installaddon}}
The following steps are optional. Installing the module as addon will enable the module to be permanently incorporated into Blender.

\noindent To install the addon:
\begin{itemize}
	\item Open Blender
	\item Open user settings (File > User preferences)
	\item Open Add-ons tab
	\item On the bottom panel press `install add-on from file`
	\item Navigate to `project folder > generator` and double click on `addon.py`
	\item In the search box on the top right corner type `NLPanim` - a greyed-out item called `Object: NLPAnim` should appear
	\item Check the box next to the item
	\item Press `Save user settings` in the bottom left corner
	\item Exit user preferences
\end{itemize}

\noindent The module is now installed as an add-on. In the 3D-view (default view) on the bottom of the right-hand side menu there should appear a tab called `Generate NLP anim` (figure ~\ref{fig:installedaddon}).
	
\begin{figure}[!ht]
	\centerline{\fbox{\includegraphics[width = 30em]{img/appendix/installedaddon.png}}}
	\caption{The generator addon menu}\label{fig:installedaddon}
\end{figure}

\section{Generating animations}
\noindent Now that all the requirements are met, the animation can be generated. There are two main steps to generating. Firstly, use the analyzer module to create a JSON file instructions for the generator. Secondly, use the generator to assemble the animation.

\subsubsection{Analyze script \label{sec:analyze-script}}
\noindent The input script must resemble the script in figure ~\ref{inputscript2}

\begin{figure}[!ht]
	\centerline{\fbox{\includegraphics[width = 30em]{img/script.png}}}
	\caption{An example of system's input}\label{fig:inputscript2}
\end{figure}

\noindent The characters must be specified after 5 tabs. The dialogue text is specified after 3 tabs. The file must end with an `ENDSCRIPT` with no indentation. For reference please refer to either:
\begin{itemize}
	\item Example script files in folder `movies`
	\item Movie scripts hosted on \url{www.imsdb.com}
\end{itemize}

\noindent The script now can be process using the following command:

\indent `python analyzer/script\_analyze.py path\_to\_dialogue\_script db/animation.db`

\noindent The program will output a file called `scene.json`. This file is used to generate the animation.

\subsubsection{Generate Animation}
\noindent If you did not follow section ~\ref{sec:installaddon}, please follow these steps:
\begin{itemize}
	\item Open the file `rendered.blend` with Blender
	\item Click `Run Script` on the bottom of the scripting view (figure ~\ref{fig:withoutaddon})
	\item The tab called `Generate NLP anim` should appear on the right hand side menu of the 3D view
	\item Click on the `Generate NLP anim` tab
\end{itemize}

\begin{figure}[!ht]
	\centerline{\fbox{\includegraphics[width = 30em]{img/appendix/withoutaddon.png}}}
	\caption{Running the generator without installing it as an add-on}\label{fig:withoutaddon}
\end{figure}

Please follow these steps only if you installed the generator as an add-on:
\begin{itemize}
	\item Open Blender
	\item Press `File > Save` and save the file in a preferred location
	\item \textbf{Important:} close Blender and reopen the saved file
	\item Open `Generate NLP anim` tab
\end{itemize}

\noindent After finishing the above instructions, do the following to finalize the animation:
\begin{itemize}
	\item Press `Clean` at the top of the menu
	\item Choose the directory `mocap` for `Animations Directory`
	\item Choose the directory `models` for `Models Directory`
	\item Choose the file `db/animation.db` for `Database`
	\item For `Dialogue` choose the `scene.json` file you created when following the section ~\ref{sec:analyze-script}
	\item press `Prepare` 
\chapter{Maintenance Manual \label{chap:maintenance}}

\section{Installation and requirements}
For information on installation and dependencies please refer to sections ~\ref{sec:requirements} and ~\ref{sec:installaddon}.

\section{Space and Memory requirements}
The project needs enough memory to install Blender and Python as well as another 100MB to fit all the animation data. Please keep in mind that rendering the animations drastically increases the memory needed as one short clip may take up to 50MB. The project comes with a few example rendered videos which take another 150MB of memory

\section{Temporary Files}
When using the EBMD importer the program will store the downloaded EBMD BVH files in the `EMBD importer/tmp' directory. These files can take a considerable amount of space and can be deleted at the convenience of a user.

\section{Files and Directories}
The files and directories are described in tables ~\ref{tab:analyzerdirectory}, ~\ref{tab:dbdirectory}, ~\ref{tab:embd-importerdirectory}, ~\ref{tab:generatordirectory} and ~\ref{tab:maindirectory}.

\begin{table}[H]
	\centering
	\small
	
	\begin{tabular}{ |p{8.5em}|p{30.1em}| }
		\hline
		\multicolumn{2}{|c|}{\textbf{Main project directory}} \\
		\hline
		\thead{Item} & \thead{Description} \\
		\hline
		analyzer/ & The directory that contains the first and second module source code (emotion analysis and matcher) \\
		\hline
		db/ & The directory that contains the sqlite database file which holds information on the animations and models \\
		\hline
		EMBD importer/ & The directory that contains the EMBD importer tool \\
		\hline
		generator/ & The directory that contains the Blender add-on \\
		\hline
		mocap/ & The directory that contains the .fbx animations \\
		\hline
		models/ & The directory that contains the .fbx character models \\
		\hline
		inputs/ & The directory that contains sample dialogue scripts \\
		\hline
		showcases/ & The directory that contains sample rendered scenes \\
		\hline
		renderer.blend & Blender project prepared for use with the system \\
		\hline
		requirements & File containing python dependencies \\
		\hline
	\end{tabular}
	
	\caption{Contents of the main directory}
	\label{tab:maindirectory}
\end{table}

\begin{table}[H]
	\centering
	\small
	
	\begin{tabular}{ |p{8.5em}|p{30.1em}| }
		\hline
		\multicolumn{2}{|c|}{\textbf{`analyzer' directory}} \\
		\hline
		\thead{Item} & \thead{Description} \\
		\hline
		db.py & The source code of the matcher module \\
		\hline
		generate\_showcase.py & The source code of the showcase generator (more information in sections ~\ref{sec:showcasegenerator} and ~\ref{sec:showcasegeneratormanual}) \\
		\hline
		parse.py & Script used to parse the dialogue script \\
		\hline
		script\_analyze.py & The main file of the analyzer program. It handles the resource pipeline. \\
		\hline
		tone.py & The script that uses the IBM Watson API to perform emotion analysis \\
		\hline
	\end{tabular}
	
	\caption{Contents of the `analyzer' directory}
	\label{tab:analyzerdirectory}
\end{table}

\begin{table}[H]
	\centering
	\small
	
	\begin{tabular}{ |p{8.5em}|p{30.1em}| }
		\hline
		\multicolumn{2}{|c|}{\textbf{`db' directory}} \\
		\hline
		\thead{Item} & \thead{Description} \\
		\hline
		animation.db & The sqlite file that holds the information on the animations and character models \\
		\hline
	\end{tabular}
	
	\caption{Contents of the `db' directory}
	\label{tab:dbdirectory}
\end{table}

\begin{table}[H]
	\centering
	\small
	
	\begin{tabular}{ |p{8.5em}|p{30.1em}| }
		\hline
		\multicolumn{2}{|c|}{\textbf{`EMBD importer' directory}} \\
		\hline
		\thead{Item} & \thead{Description} \\
		\hline
		imported/ & The directory that holds the imported animations \\
		\hline
		tmp/ & The directory that caches downloaded EMBD animations \\
		\hline
		armature.fbx & Character armature used for reference during importing. This armature is used for all the animations in the project \\
		\hline
		importer.blend & Blender project prepared to handle importing animations from EMBD \\
		\hline
		script.py & Source code of the importer. This script is meant to be run from inside of Blender \\
		\hline
	\end{tabular}
	
	\caption{Contents of the `EMBD importer' directory}
	\label{tab:embd-importerdirectory}
\end{table}

\begin{table}[H]
	\centering
	\small
	
	\begin{tabular}{ |p{8.5em}|p{30.1em}| }
		\hline
		\multicolumn{2}{|c|}{\textbf{`generator' directory}} \\
		\hline
		\thead{Item} & \thead{Description} \\
		\hline
		addon.py & The source code of the generator module \\
		\hline
	\end{tabular}
	
	\caption{Contents of the `generator' directory}
	\label{tab:generatordirectory}
\end{table}


\section{Known Issues}
\noindent Please make sure that the dialogue script is correctly formatted before running the emotion analysis. Wrong indentation levels may cause unpredictable results.

\noindent The generator module will provide you with an error message if the input (scene.json) is malformed. However if the file parses correctly but it's data is corrupted it may lead to unpredictable results. This situation will never take place is the scene.json file was generated by the script\_analyze.py program.



\chapter{Other \label{chap:otherappendices}}


\section{Useful Links}
Sample generated animations and animations used in the questionnaire: \videoshost \newline
The Emotional Body Motion Database: \url{http://ebmdb.tuebingen.mpg.de/} \newline
IBM Watson Tone Analyzer: \url{https://www.ibm.com/watson/services/tone-analyzer/} \newline
Tone Analyzer Guide: \url{https://console.bluemix.net/docs/services/tone-analyzer/} \newline
Blender: \url{http://www.blender.org/} \newline
International Movie Script DataBase: \url{http://www.imsdb.com} \newline
DB browser for Sqlite: \url{ttp://www.sqlitebrowser.org} \newline


\section{The questionnaire \label{sec:appquestionnaire}}
Please see figures ~\ref{fig:firstquestionnaire} to ~\ref{fig:lastquestionnaire}.


\begin{figure}[h]
	\makebox[\textwidth][c]{\includegraphics[page=1,width=1.2\textwidth]{img/appendix/questionnaire.pdf}}
	\caption{The questionnaire - page 1}\label{fig:firstquestionnaire}
\end{figure}

\begin{figure}[h]
	\makebox[\textwidth][c]{\includegraphics[page=2,width=1.2\textwidth]{img/appendix/questionnaire.pdf}}
	\caption{The questionnaire - page 2}
\end{figure}

\begin{figure}[h]
	\makebox[\textwidth][c]{\includegraphics[page=3,width=1.2\textwidth]{img/appendix/questionnaire.pdf}}
	\caption{The questionnaire - page 3}
\end{figure}

\begin{figure}[h]
	\makebox[\textwidth][c]{\includegraphics[page=4,width=1.2\textwidth]{img/appendix/questionnaire.pdf}}
	\caption{The questionnaire - page 4}
\end{figure}

\begin{figure}[h]
	\makebox[\textwidth][c]{\includegraphics[page=5,width=1.2\textwidth]{img/appendix/questionnaire.pdf}}
	\caption{The questionnaire - page 5}
\end{figure}

\begin{figure}[h]
	\makebox[\textwidth][c]{\includegraphics[page=6,width=1.2\textwidth]{img/appendix/questionnaire.pdf}}
	\caption{The questionnaire - page 6}\label{fig:lastquestionnaire}
\end{figure}



\section{Sample generated scene used in the questionnaire \label{sec:samplescene}}
The following scene comes from the game \textit{Horizon: Zero Dawn}. The screenshots depicting the scene can be found in figures ~\ref{fig:screenshotfirst} to ~\ref{fig:screenshotlast}. Please note that the top images represent the original scene while the bottom ones show the recreated scene. 


\begin{figure}[h]
	\centerline{\fbox{\includegraphics[width = 30em]{img/appendix/sample/horizon1.png}}}
	\centerline{\fbox{\includegraphics[width = 30em]{img/appendix/sample/horizonrender1.png}}}
	\caption{Dialogue line 1. \\ Transcript: `Aloy: Are you sure you're okay?'}\label{fig:screenshotfirst}
\end{figure}

\begin{figure}[h]
	\centerline{\fbox{\includegraphics[width = 30em]{img/appendix/sample/horizon2.png}}}
	\centerline{\fbox{\includegraphics[width = 30em]{img/appendix/sample/horizonrender2.png}}}
	\caption{Dialogue line 2. \\ Transcript: `Erend: I'm sober enough, all right? I don't need another lecture!'}
\end{figure}

\begin{figure}[h]
	\centerline{\fbox{\includegraphics[width = 30em]{img/appendix/sample/horizon3.png}}}
	\centerline{\fbox{\includegraphics[width = 30em]{img/appendix/sample/horizonrender3.png}}}
	\caption{Dialogue line 3. \\ Transcript: `Aloy: That's not what I meant, I was talking about what happened outside with the crowd.'}
\end{figure}

\begin{figure}[h]
	\centerline{\fbox{\includegraphics[width = 30em]{img/appendix/sample/horizon4.png}}}
	\centerline{\fbox{\includegraphics[width = 30em]{img/appendix/sample/horizonrender4.png}}}
	\caption{Dialogue line 4. \\ Transcript: `Erend: I don't want to talk about that. We're here because of what you said about Olin, so do what you need to do.'}\label{fig:screenshotlast}
\end{figure}


\section{Interviewees' comments about the recreated animations \label{sec:qcomments}}
The following is a full list of comments that the interviewees made when asked to justify some of their choices. Those comments were not required of the participants and some refused to give more detailed feedback. Please note that Animation/Option/Video 1 refers to the original animation, while the Animation/Option/Video 2 refers to the recreated one.

\noindent Comments in favour of the recreated animations:
\begin{itemize}
	\item `The second animation is corresponding better to the situation in the game.'
	\item `Active body language refers to what is being said.'
	\item `Better expression of emotions but sometimes they were exaggerated.'
	\item `Both animations present natural movements of the body.'
	\item `No body language [in animation 1].'
	\item `More accurate expression of feelings.'
	\item `Option 2 because character has more natural behavior.'
	\item `Body language matches with what is being said.'
	\item `Natural body motions [in animation 2].'
\end{itemize}

\noindent Comments in favour of the original animations:
\begin{itemize}
	\item `Weird hands movement in the 2nd one.'
	\item `Option 1 because in option 2 author moved too far.'
	\item `Weird hand movement in the 2nd one.'
	\item `[Animation 1] Looks better.'
\end{itemize}


\noindent Other comments:
\begin{itemize}
	\item `Both about the same, animation 1 didnt represent any emotion, animation 2 was little over the top'
	\item `Both animation represented the scene about the same, didnt really have a preference.'
	\item `I think both represented the emotion about the same.'
	\item `Option 1 is not bad, option 2 is overly exaggerated.'
\end{itemize}





\end{document}
